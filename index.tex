\documentclass{article}
\usepackage[margin=1in]{geometry} 
\usepackage{amsmath,amsthm,amssymb}
\usepackage{algorithm}
\usepackage{algorithmic}
% \newcommand{\R}{\mathbb{R}}
\newcommand{\N}{\mathbb{N}}
\newcommand{\Z}{\mathbb{Z}}
\newcommand{\F}{\mathbb{F}}
\newcommand{\C}{\mathbb{C}}
\newcommand{\D}{\mathbb{D}}
% \newcommand{\<}{\langle}
% \renewcommand{\>}{\rangle}
% \newcommand{\innerproduct}[1]{\< #1 \>}
\newcommand{\abs}[1]{\left\vert #1 \right\vert}
\newcommand{\norm}[1]{\left\Vert #1 \right\Vert}
\renewcommand{\Re}{\text{Re}}
\renewcommand{\Im}{\text{Im}}
% \newcommand{\rank}{\text{rank}}
\newcommand{\Part}[2]{\frac{\partial #1}{\partial #2}}
% \renewcommand{\span}{\text{span}}
% \newcommand{\tr}{\text{tr}}
\newcommand{\argmin}{\mathop{\arg\min}}
\newcommand{\argmax}{\mathop{\arg\max}}
% \newcommand{\ria}[2]{#1 \rightarrow #2}
\newcommand{\etal}{\textit{et al. }}
% \providecommand{\keywords}[1]{\textbf{\text{Index terms---}}#1}


\begin{document}
\title{Idea summary for HBS segmentation project}
\author{Lin Chenran}
\maketitle

Our goal is to build a quasi-comformal segmentation model with shape prior. HBS is a powerful signature of simply-connected shapes for this task. We will use the model to segment the image into 2 parts, target shape and background, and the HBS can refine the segmentation result.


\section{Model 1}
At begining, we will use a simple model to test the idea as
\begin{equation}\label{eq1}
    E(f) = \int_\Omega | I \circ f - J  |^2 + \alpha \int_\Omega | \nabla f |^2 + \beta \int_\Omega | \mu(f) - \bar{\mu} |^2
\end{equation}
where $f$ is the segmentation function, $I$ is the image, $J$ is the template shape, $\mu(f)$ is the Beltrami coefficient of $f$, $\bar{\mu}$ is the shape prior, $\alpha$ and $\beta$ are the weights of the regularization terms. Actually, the shape prior $\bar{\mu}$ is the mean of HBS of a special kind of shapes. The first item of (\ref{eq1}) is the difference between deformed template and the given image. The second item is the smoothness term. The third item is the shape prior term.

To optimize the model, we use variable split method to release the contrain between $f$ and $\mu(f)$, which should satisify Beltrami equation. The model (\ref{eq1}) can be rewritten as
\begin{equation}\label{eq2}
    E(f, \nu) = \int_\Omega | I \circ f - J  |^2 + \alpha \int_\Omega | \nabla f |^2 + \beta \int_\Omega | \nu - \bar{\mu} |^2 + \gamma \int_\Omega | \nu - \mu(f) |^2.
\end{equation}
Then we split it into 2 subproblems to minimize (\ref{eq2}), which is $f$-subproblem
\begin{equation}\label{eq3}
    f_{n+1} = \arg\min_{f} \left\{ \int_\Omega | I \circ f - J_n  |^2 + \alpha \int_\Omega | \nabla f |^2 \right\},
\end{equation}
and $\nu$-subproblem
\begin{equation}\label{eq4}
    \nu_{n+1} = \arg\min_{\nu} \left\{\beta \int_\Omega | \nu - \bar{\mu} |^2 + \gamma \int_\Omega | \nu - \mu_{n+1} |^2 \right\},
\end{equation}

\subsection{Initialization step}
There two main changes of the model comparing with traditional segmentation model. First, we use HBS as the shape prior by adding item $|\mu(f) - \bar{\mu} |$. Second, we start the iteration from the given shape prior $\bar{\mu}$.

In the initialization step, we use the given HBS $\bar{\mu}$ to reconstruct the initial function $\bar{f}$. Because of the properties of HBS, $\bar{f}$ is only defined on $\mathbb{D}$ and can only be determined up to a rotation, translation and scaling. Let $T(z) = az +b $, then we can find the optimal $T_0$ by solving the following problem 
\begin{equation}\label{eq5}
    T_0 = \arg\min_T \int_\Omega | I \circ T \circ \tilde{f}  - J |^2,
\end{equation}
where $\tilde{f} = \bar{f}$ on $\mathbb{D}$ and $\tilde{f} = Id$ on $\mathbb{C} \setminus \mathbb{D}$. Then we can get the initial function $f_0 = T_0 \circ \tilde{f}$ and $J_0 = J \circ f_0^{-1}$.

Now we start the iteration from $f_0$. In each iteration, we first solve the $f$-subproblem (\ref{eq3}) to get $f_{n+1}$, then solve the $\nu$-subproblem (\ref{eq4}) to get $\nu_{n+1}$. We repeat the iteration until the energy $E(f_n, \nu_n)$ converges.

\subsection{f-subproblem}
Let $f = Id + u$, consider the Taylor expansion of energy function of (\ref{eq3})
\begin{equation}
    E_f \approx \| \nabla I (f - Id ) + I - J_n \|^2_2 + \alpha \| \nabla f \|^2_2 = \| \nabla I u + I - J_n \|^2_2 + \alpha \| \nabla u \|^2_2.
\end{equation}
The Euler-Lagrangian equation $\partial E_f / \partial u = 0$ for minimizing the above energy functional shows that the minimizer $u_{n+1}$ satisfies
\begin{equation}
    ( \nabla I ^2 - \alpha \Delta) u_{n+1} = \nabla I (J_n - I),
\end{equation}
which is a linear equation of a vector function and can be solved easily, by incorporating with the boundary condition $u_n+1 =0 $ on $\partial \Omega$.

After getting $f_{n+1} = u_{n+1} + Id$, we can update $J_{n+1} = J_n \circ f_{n+1}^{-1}$, $F_{n+1} = f_{n+1} \circ F_n$ and then $\mu_{n+1} = \mu(F_{n+1})$.

\subsection{$\nu$-subproblem}
The energy function of (\ref{eq4}) is
\begin{equation}
    E_\nu = \beta \| \nu - \bar{\mu} \|^2_2 + \gamma \| \nu - \mu_{n+1} \|^2_2.
\end{equation}
Therefore, the minimizer is
\begin{equation}
    \nu_{n+1} = \frac{\beta \bar{\mu} + \gamma \mu_{n+1}}{\beta + \gamma}.
\end{equation}

\subsection{Summary of algorithm}

\begin{algorithm}[H]                           % HERE!!!!!!!!!
        \caption{Algothim 1}          % give the algorithm a caption
        \label{alg1}      % and a label for \ref{} commands later in the document
        \begin{algorithmic}  % enter the algorithmic environment
            \STATE \textbf{Inputs:} Image $I$, template $J$, HBS $\bar{\mu}$.
            \STATE \textbf{Initialize:} reconstruct $\bar{f}$ from $\bar{\mu}$, then extend to $\tilde{f}$.
            \STATE Find $T_0$ by solving (\ref{eq5}), and let $f_0 = T_0 \circ \tilde{f}$, $J_0 = J \circ f_0^{-1}$, $F_0 = f_0$, $n=0$.
            \WHILE{$n < N$}
                \STATE Solve $f$-subproblem (\ref{eq3}) to get $f_{n+1}$.
                \STATE Let $\mu_{n+1} = \mu(f_{n+1} \circ F_n)$.
                \STATE Solve $\nu$-subproblem (\ref{eq4}) to get $\nu_{n+1}$.
                \STATE Reconstruct $f^{\nu}_{n+1}$ from $\nu_{n+1}$.
                \STATE Let $J_{n+1} = J_n \circ (f^{\nu}_{n+1})^{-1}$ and $F_{n+1} = f^{\nu}_{n+1} \circ F_n$.
                \IF{converges}
                    \STATE \textbf{break}
                \ENDIF
                \STATE Let $n = n+1$.
            \ENDWHILE
            \RETURN Segmentation function $F_{n+1}$.
        \end{algorithmic}
        \end{algorithm}

\section{Model 2}
To be continued...
\begin{equation}
    E(T, H) = \int_\Omega | I - J \circ T \circ g_H \circ H |^2 + \alpha \int_\mathbb{D} | \Delta H |^2 + \beta \int_\mathbb{D} | \mu(H) - \bar{\mu} |^2
\end{equation}

\end{document}